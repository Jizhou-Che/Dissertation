\documentclass[a4paper,12pt,twoside]{article}

\usepackage{proposal}

\begin{document}

\title{Project Proposal (Draft):\\Ontology-based Data Integration}
\author{Jizhou Che}
\submitdate{October 2020}
\degree{BSc Computer Science with Artificial Intelligence}
\studentid{20032291}
\studentemail{scyjc1@nottingham.edu.cn}
\supervisor{Dr. Heshan Du}

\normallinespacing
\maketitle

% Delete the two declaration sentences in proposal.sty if not applicable. 

% \preface

% \addcontentsline{toc}{chapter}{Abstract}

\begin{abstract}

Unlike schema-rich knowledge bases which focus a lot on the definitions of terminologies and their relationships, a typical dataset records numerous data entries which fall into those terminologies. In an ontology, such terminologies are represented as classes, and data entries are represented as instances of those classes. To perform integration between datasets, both the schema and the data entries should be matched together, which can be done by matching the classes and instances from their respective ontologies. While matching systems for classes have evolved a lot in recent years, instance matching remains immature. This project investigates into the additional information that can be extracted from ontologies, and explores the possibility of using a reasoning-based approach to automatically deduct the relationships between ontology entities from those information. An automated reasoning-based ontology matching system was built and evaluated using real-world ontologies and reference alignments, including official ones used in the ontology alignment evaluation campaigns. Precision and recall rates were calculated and compared with major ontology matching systems. The overall evaluation results showed great matching precision and satisfying recall, F-measure and time complexity. This system can serve as a basement for real-world data integration tasks.

\end{abstract}


% \input{tex/acknowledgements}

% There is a maximum limit of 15,000 words without exceeding 40 pages (A4 sides) for the main body of the dissertation that will be submitted in PDF. This limitation includes the bibliography and excludes cover/front pages (e.g., abstract, acknowledgement, table of contents) and excludes the appendices, listing of any codes or any other supporting documentation.
% Note: Your dissertation should not exceed the word and page limits. You do not have to use up your word limit to get a good grade; never `pad out' your dissertation, this will only annoy the markers.

% \body

% \chapter{Introduction}

This chapter introduces the importance and challenges of data integration, and provides a summary of Semantic Web technologies to demonstrate the feasibility of integrating datasets by instance matching based on ontologies. This chapter also states the motivation and aims of this project to formalise its research focus.


\section{Background}

% Data integration: demand, application, importance, benefits; Ontology matching in data integration;
The ability of using data from multiple sources simultaneously has always been demanded in both research and public areas. For example, translational research data from multiple biomedical domains can be used together to track and analyze human-centered records \cite{DBLP:journals/jbi/WangLFCOHSO09}, and data exported from each peer in its own schema need to be combined in a peer-to-peer (P2P) system \cite{DBLP:conf/dbisp2p/CalvaneseDGLR03}. The process of data integration seeks to identify the semantic interconnections between information from different sources, to promote the seamlessness and effectiveness of operations across multiple datasets in such applications.
\\\\
% The semantic heterogeneity problem of data integration.
Traditional data integration has been problematic due to the semantic heterogeneity problem of datasets, which can be reflected in various aspects. Firstly, the accessibility and usability of data are unpromising, as resources are usually managed by different organisations and stored in different formats such as Excel or scanned PDF, which are not easily readable to computer programmes. In addition, without a formal representation of the semantics of the knowledge within the datasets, computer programmes can only interpret the information at a very basic level \cite{DBLP:journals/expert/ShadboltBH06}, meaning that reasoning about the data can hardly be conducted. Further more, the matching process needs to be carried out even if the datasets come from the same domain of interest, due to the lack of a uniform data modelling principle and a shared vocabulary \cite{euzenat2013d}. Integration of such datasets requires huge amount of extra effort and can sometimes be unrealistic.
\\\\
% Semantic Web foundations and technologies.
The idea of Linked Data and Semantic Web provides a fundamental framework for addressing these issues. Based on the Uniform Resource Identifiers (URIs) \cite{DBLP:journals/rfc/rfc3986} and the HyperText Transfer Protocol (HTTP) \cite{DBLP:journals/rfc/rfc7540} technologies, the idea of Linked Data was formulated so that entities in the world can be uniquely identified by URIs and dereferenced with HTTP. This provides a common way to retrieve entity descriptions from the network \cite{DBLP:journals/ijswis/BizerHB09}. This framework is supplemented by the Resource Description Framework (RDF), which provides a universal, graph-based structure for the data to be modelled, and also to be queried with the SPARQL Protocol and RDF Query Language (SPARQL) \cite{DBLP:journals/tods/PerezAG09}. A shared vocabulary of a knowledge domain can be established as an ontology that is defined in the Resource Description Framework Schema (RDFS) or Web Ontology Language (OWL) \cite{DBLP:journals/expert/ShadboltBH06}, while the entires in one vocabulary can be linked to those in another with RDF triples or OWL axioms \cite{DBLP:journals/ijswis/BizerHB09}. Rules for inference purposes can be specified with the help of Rule Markup Languages (RuleML) \cite{DBLP:journals/entcs/MeiB06}.
\\\\
% Ontology matching.
With the advances of formal knowledge representation with ontologies in the context of Semantic Web, ontology matching techniques have evolved to take advantage of structured domain knowledge and assist in data integration tasks. Information coming from multiple sources are integrated without producing a newly merged dataset. Instead, they are captured by individual ontologies developed to model their respective knowledge domains, and between the ontologies, corresponding classes and instances are linked together to produce an alignment (Figure \ref{fig:align}). This alignment can be used to generate a mediator that can perform various tasks such as data translation, query reformulation and ontology merging \cite{euzenat2013d}, meaning that a query composed over a single dataset can now be reformulated as multiple queries over all the datasets, hence achieving the goal of data integration. This approach has become one of the most recognised solutions to the semantic heterogeneity problem in data integration \cite{DBLP:conf/gil/NafissiBF18}.
\\\\
\begin{figure}[ht]
\begin{center}
\includegraphics[width=0.3\textwidth]{img/ontomatch.png}
\caption{Alignment production}
\label{fig:align}
\end{center}
\end{figure}
\\\\
% Instance matching.
The general term \textquotedblleft ontology matching\textquotedblright \space involves the matching of both terminological components (TBox), which are related to the classes or concepts, and assertional components (ABox), which are related to the instances or individuals. The concepts of TBox and ABox come from Description Logic (DL) \cite{DBLP:books/daglib/0041477}, which is the mathematical foundation of ontology languages \cite{DBLP:journals/cacm/Horrocks08}. Considering the application in data integration tasks, while the matching of TBoxes (schema-level matching) is important, data are not always presented with proper schema specification \cite{DBLP:conf/boemie/CastanoFMV11}. Further more, the discovery of corresponding instances is necessary for further reasoning on the instance level. This has resulted in a gradual shift of research interests into the instance matching techniques in the recent years, and the introduction of such techniques have shown great achievements and potentials in popular matching systems \cite{DBLP:conf/scdm/AbubakarHMA18}.

% The need of using information simultaneously from multiple sources has been dramatically increasing in the recent researches [? citation] \cite{placeholder}. For example, datasets of different urban infrastructure assets can be integrated to produce a decision support system for city management [heshan], and --- []. Unfortunately, resources are usually constructed and maintained independently in heterogeneous formats [? citation] \cite{placeholder}. The traditional approach of making use of the interdependencies between datasets relies on matching the semantically related entities at the schema level [Ontology matching, 8; Batini et al. 1986; Sheth and Larson 1990; Spaccapietra and Parent 1991; Parent and Spaccapietra 1998] \cite{placeholder}. An obvious drawback of this approach is that updates performed on the individual datasets are not reflected in the merged dataset without extra work. Further more, the matching process needs to be carried out even if the datasets come from the same domain of interest, due to the lack of a uniform modelling principle [ontology matching, 8]\cite{placeholder}.
% \\\\
% Another challenge resides in that the usability of data in the traditional web is usually questionable: datasets can be represented in formats such as CSV, Excel, HTML, PDF table or even scanned pictures [Ontology matching, 11, linked data]. Computers play a very limited role in interpreting such media, doing solely words indexing and document serving, while the intelligent works such as selecting, combining and aggregating the data are performed by human[citation: A Semantic Web Primer, series foreword, xvi].
% \\\\
% The traditional solutions to dataset integration typically relies on matching the semantically related entities at the schema level [Ontology matching, 8; Batini et al. 1986; Sheth and Larson 1990; Spaccapietra and Parent 1991; Parent and Spaccapietra 1998] or the catalog level [Bouquet et al. 2003b, Bernstein and Rahm 2000]. The drawback of such approaches is obvious: updates on the individual datasets are not reflected in the merged dataset without extra work. Further more, the matching process needs to be carried out even if the datasets come from the same domain of interest, due to the lack of a uniform data modelling principle [ontology matching, 8].
% \\\\
% Based on the concept of the semantic web, the research of data integration propose the approach where integration of information coming from multiple local sources is performed without first loading the data into a central warehouse [Chawathe et al. 1994; Wache et al. 2001; Draper et al. 2001; --Halevy et al. 2005--; Seligman et al. 2010; Doan et al. 2012;]\cite{placeholder}. First, a common ontology need to be built as an inter-connection between the ontologies of the domains where the data are sourced from [Ontology matching, 10].
% \\\\
% Description logics (DLs) are a family of knowledge representation languages that can be used to represent knowledge of an application domain in a structured and well-understood way [Citation: An Introduction to Description Logic, 1] \cite{placeholder}. Recently, DLs have played a central role in the semantic web [Hor08], having been adopted as the basis for ontology languages [HPSvH03]. They typically separate domain knowledge into two components: the terminological part TBox and the assertional part ABox, where the combination of a TBox and an ABox is called a knowledge base (KB) [Citation: An Introduction to Description Logic, 1] \cite{placeholder}. A crucial feature of DLs is that terminological and assertional statements have a formal logic-based semantics, meaning that entailments of such statements in a knowledge base can be determined through automated reasoning [Citation: An Introduction to Description Logic, 1] \cite{placeholder}.


\section{Motivation}

A knowledge base typically contains not only a rich schema, but also a large number of instances with enormous assertions about them. For the task of data integration, the alignment of matching instances should be emphasised and produced in a manner that utilises both schema-level and instance-level information. Although a lot of researches have been conducted to propose various kinds of instance matching techniques, most of them do not use sufficient information at the instance level. Specifically, the role axioms that demonstrate properties of relations between individuals, often referred to as the RBox in a description logic context, are often neglected. For example,

\begin{spacing}{1.2}
$$hasMother \sqsubseteq hasParent$$
$$hasParent \sqsubseteq hasAncestor$$
$$Trans(hasAncestor)$$
$$Asym(childOf)$$
$$Disj(hasSibling, childOf)$$
\end{spacing}

are some of the role axioms concerning family relationships that can appear in an ontology.
\\\\
The research focus of this project is to develop an instance matching algorithm that integrates the matching of such role axioms into existing techniques. It is believed that this will benefit the discovery of implicit instance-level knowledge through automated logical reasoning, hence improve the precision and reliability of instance matching for the purpose of data integration.


\section{Aims and Objectives}

The aim of this project is to examine the proposed instance matching idea that integrates the matching of role axioms, by designing and implementing the algorithm to deliver a software tool as a tangible result that can be used for evaluation. In order to achieve this, the following objectives need to be reached:

\begin{spacing}{1.2}
\begin{enumerate}
	\item Study the fundamentals of Semantic Web tachniques and ontology representations.
	\item Explore and manipulate existing strategies for ontology matching and instance matching.
	\item Design the instance matching algorithm to integrate the matching of roles. Analysation of existing approaches will be conducted to facilitate the shaping of the new design.
	\item Implement the design and wrap it in an evaluation framework, to produce a tangible data integration software.
	\item Test and evaluate the implementation, and potentially use description logic reasoners to reason about the implementation.
\end{enumerate}
\end{spacing}

This project is interested in producing the alignment in a fully automatic manner, and tries to avoid manual parameter settings. It should also be noted that it is not the target of the project to achieve complete correctness, which is considered impractical without the aid of domain expert knowledge. However, the level of precision and completeness is still considered as one of the most critical evaluation criterion. The underlying reasonability and decidability of the logic will also be taken into account.


\section{Interim Report Outline}

This chapter has summarised the general background of Semantic Web technologies and the idea of ontology-based instance matching for data integration. The motivation and aims of this project has been stated. Chapter 2 provides a detailed description of previous works related to ontology matching, specifically instance matching, and mentions the most popular matching systems at the present. Chapter 3 justifies the selection of methodologies for the design, and propose a structure of integrating them. Chapter 4 then demonstrates the tools and frameworks used in the implementation steps. Chapter 5 presents the project management and reflections along the progress.


% \chapter{Related Work}

This chapter outlines the major studies in ontology matching methodologies and specifically targets the field of instance matching, by providing a high-level classification with respect to their main characteristics. Based on these methods, the structures of the most recent and impactful matching systems are introduced along with their latest evaluation results.


\section{Classification of Techniques}

Many approaches have been proposed for the classification of ontology matching techniques. For example, one way is to look at the input, process and output dimensions of the algorithms \cite{euzenat2013d}. Characteristics such as the representation models accepted as input, the type of context used, and the cardinality of output alignments are taken into account. Another way is to distinguish between local methods that directly target the properties of the entities, such as string comparison, and globals methods that investigate into the structures that hold the entities together, such as graph segments. The most accepted classification method was proposed by Jérôme Euzenat and Pavel Shvaiko, which considers the different sources and interpretations of the input information, and further classify them with respect to how they are being processed \cite{euzenat2013d}. Other classification methods from different inspirations exist \cite{DBLP:conf/swb/Ehrig2007,DBLP:conf/icde/MadhavanBDH05}, but the ideas are generally similar.
\\\\
Based on the fundamental differences identified in the above mentioned approaches, this section describes a classification method focusing on the algorithmic characteristics of the local methodologies used for ontology matching. The proposed classification is shown in Figure \ref{fig:techniques}.

\begin{figure}[ht]
\includegraphics[width=\textwidth]{img/ontology_matching_classification.png}
\caption{Classification of Ontology Matching Techniques}
\label{fig:techniques}
\end{figure}

\subsection{Similarity-based}

Similarity-based techniques try to perform a quantitative measurement on the degree of similarity of two entities. Generally speaking, entities are concrete ontological components such as classes, datatypes, properties, and named individuals, and while the main focus is in the context of instance matching, all of them should be considered to facilitate the design of the whole structure. Linguistic similarity and contextual similarity are the two most-used criteria for the measurement \cite{DBLP:conf/ic3k/Nguyen015}.

\subsubsection{Linguistic Similarity}

The linguistic components of an instance are the characteristics related to its string representation. For similarity measurements of strings, syntactic analysis and semantic analysis are the two main approaches.
\\\\
% Syntactic.
The syntactical components of an instance can be the string structure of its name and description, its type constraints, the list of properties it possesses, and so on. Syntactic techniques take advantage of these string structures to discover different kinds and levels of similarities. For example, measures like Edit Distance, Smith-Waterman Distance, and Jaro Distance are based on characters \cite{DBLP:journals/corr/SekerAAM14,DBLP:journals/algorithmica/FaroMP20}, while measures like Cosine Similarity and Q-Gram Distance are based on tokens \cite{DBLP:conf/wea/KobayashiHYS20}. Different weights can be introduced according to the occurrence frequency of the values, so that matches of rare values can receive higher confidence \cite{DBLP:conf/dsit/WangN20}. Further more, phonetic similarities can be discovered with techniques like Soundex \cite{DBLP:journals/corr/BhattiWIHS14} and Metaphone \cite{DBLP:conf/cicling/JordaoR12}, even though their string representations are very different.
\\\\
% Semantic.
Semantic approaches, in contrast, seek to analyse the actual meaning of the entities to be matched, either by informal methods such as looking up dictionaries or thesauri and reading the annotations \cite{DBLP:journals/corr/abs-0907-2209,DBLP:conf/ifip12/LinS08,DBLP:conf/esws/Vennesland15}, or formal methods such as deriving the meaning from semantic definition in other ontologies. Techniques based on this approach first expand the meanings of entities to be matched with synonyms and hyponyms, then evaluate the intersection of their meanings to perform similarity calculations \cite{DBLP:conf/ieaaie/AssiMD19}.

\subsubsection{Contextual Similarity}

Context-based techniques try to match the instances by considering not only their personal attributes, but also how they are defined in their respective ``contexts'', such as object properties and semantic relations with other instances \cite{DBLP:journals/jucs/LinSX12}. The context of an entity can typically be represented as a graph, where entities are denoted by nodes, and properties and semantic relations are denoted by edges. Graph matching algorithms can then be applied to evaluate the topological similarity between the context graphs, therefore determine the similarity between those instances \cite{DBLP:conf/kdd/JehW02}.

\subsection{Reasoning-based}

A fundamental difference between reasoning-based techniques and similarity-based ones is that, in order that reasonings can start, the existence of an initial set of alignments is required as a foundation \cite{DBLP:conf/esws/Meilicke09}. These alignments can be semantic relations already existing between the two ontologies, or manually established by domain experts, or acquired automatically through some similarity-based techniques. Based on these initial alignments, implications of new alignments can be derived by performing various kinds of reasoning tasks. The process goes on iteratively based on the new alignments discovered in the previous iterations, until a time limit or a fixed number of iterations is reached, or no new alignments can be discovered \cite{DBLP:journals/logcom/MeilickeST09}.

\subsubsection{Deductive Reasoning}

Most of the reasoning techniques are considered deductive, as they typically use implication models in the derivation of new candidate alignments. The propositional satisfiability (SAT) model \cite{DBLP:journals/ai/PhamTS08}, for example, treats a candidate alignment between two instances as a propositional formula or implication, whose satisfiability is checked with SAT solvers to determine the correctness of the candidate alignment. Another widely adopted deductive approach is based on the description logic model, in which case existing alignments are used together with axioms in the respective ontology such as subsumptions, to exploit and infer new alignments to enrich the initial set \cite{DBLP:phd/ethos/Reul12}.

\subsubsection{Probabilistic Reasoning}

Probabilistic reasoning approaches, on the other hand, try to compute the probability that two entities are identical or similar, by checking whether they have similar hierarchical structure or semantic relations \cite{DBLP:conf/esws/CastanoFLNM08}. Most of them utilise heuristic algorithms \cite{DBLP:conf/jist/NguyenI15}, machine learning techniques or Bayesian networks \cite{DBLP:conf/semweb/SvabS06} for the calculation and propagation of joint probabilities. The interest in techniques that adopt machine learning have been increasing in recent years, however the difficulty in finding a good training dataset is still challenging. Other related techniques based on data analysis and statistical techniques include fuzzy ontologies \cite{DBLP:conf/semweb/TodorovHG13}.


\section{Matching Systems}

A number of matching systems have been proposed in this area, some for general purposes and the other target specific use cases. The list below selects the ones that are most impactful and have distinctive characteristics and gives a brief description of the methodologies they employ \cite{euzenat2013d}.

\begin{spacing}{1.2}
\begin{enumerate}
	\item H-Match: Computes linguistic and contextual similarities and then combine them with weighting schemas to give the final measure of semantic affinity. Common thesauri are used and extended to consider taxonomical or mereological relations.
	\item COMA++: A schema matching tool that utilises string-based techniques, auxiliary thesauri and parallel composition of matchers. Semantically it supports alignment reuse, and structurally it promotes Directed Acyclic Graph (DAG) matching.
	\item MapOnto: A semi-automated system that constructs complex mapping formulas in Horn clauses. Simple external alignments are taken as part of the input, and structural comparison between the graphical representation of input ontologies are performed.
	\item CtxMatch: Besides the semantic approaches based on strings and WordNet, it exploits description logic reasoners such as Pellet \cite{DBLP:journals/ws/SirinPGKK07} and FaCT \cite{DBLP:conf/cade/TsarkovH06}.
	\item S-Match: Semantic matching is performed based on strings and languages. Entity relations are then translated into propositional formulas using the WordNet dictionary, and resolved with SAT solvers.
	\item Lily: Combines string-based techniques such as Levenshtein Distance and structure-based techniques such as variations of Similarity Flooding, to handle large ontologies.
	\item OLA: Compiles the input ontologies into graph structures, then computes string-based, language-based and structure-based similarities. An iterative fixed point algorithm is then performed until no improvements can be discovered.
	\item RiMOM: Uses Bayesian decision theory to automatically discover complex alignments. Taxonomic structure similarities are heuristically discovered and then propagated.
	\item LogMap: Performs lexical indexing using string-based methods and with the help of WordNet. The tree structures of indices are compared, and the matches are repaired using the Dowling-Gallier algorithm based on propositional Horn satisfiability.
	\item PARIS: A probabilistic system that aligns relations, instances and schemas via iterative fixed point computation.
\end{enumerate}
\end{spacing}

It can be discovered that most of the matching systems are schema-based, and tend to focus on specific application domains. Only few of them target the general problem, such as COMA++ and S-Match. Another concern is that most of them handle tree structures while only COMA++ and OLA handle graphs. Further more, only few of them can discover complex correspondences other than the simple one-to-one. And Finally, most of them lack a graphical user interface \cite{euzenat2013d}.
\\\\
The Ontology Alignment Evaluation Initiative (OAEI)\footnote{\url{http://oaei.ontologymatching.org/}} is a coordinated international initiative to assess strengths and weaknesses of alignment or matching systems, and to compare performance of techniques. It has been organising ontology matching workshops since 2004. Tables \ref{tab:SPIMBENCH_SANDBOX_2020}, \ref{tab:SPIMBENCH_MAINBOX_2020}, \ref{tab:Knowledge_Graph_2020_Overview}, \ref{tab:Knowledge_Graph_2020_Instance_Results} shows the most recent results in 2020 for the SPIMBENCH and Knowledge Graph tracks for the matching systems participated\footnote{\url{https://project-hobbit.eu/challenges/om2020/}}. The reason of considering only these results is that those evaluation tracks focus specifically on instance matching. It can be seen that Lily has achieved excellent precision and recall rate within acceptable time frame, while AML and ATBox outperforms the others in instance matching for Knowledge Graph.

\begin{table}[ht!]
\resizebox{\textwidth}{!}{%
\begin{tabular}{|l|l|l|l|l|l|}
\hline
System          & LogMap-HOBBIT & AgreementMakerLight-Hobbit & Lily\_HOBBIT-2.2 & FTRLIM      & REMiner-1.5 \\ \hline
Fmeasure        & 0.841328413   & 0.864516129                & 0.991708126      & 0.921417565 & 0.998324958 \\ \hline
Precision       & 0.938271605   & 0.834890966                & 0.983552632      & 0.854285714 & 1           \\ \hline
Recall          & 0.762541806   & 0.89632107                 & 1                & 1           & 0.996655518 \\ \hline
timePerformance & 7483          & 6446                       & 2050             & 1525        & 7284        \\ \hline
\end{tabular}%
}
\caption{SPIMBENCH SANDBOX 2020}
\label{tab:SPIMBENCH_SANDBOX_2020}
\end{table}

\begin{table}[ht!]
\resizebox{\textwidth}{!}{%
\begin{tabular}{|l|l|l|l|l|l|}
\hline
System          & LogMap-HOBBIT & AgreementMakerLight-Hobbit & Lily\_HOBBIT-2.2 & FTRLIM      & REMiner-1.5 \\ \hline
Fmeasure        & 0.785635764   & 0.860457622                & 0.995388669      & 0.921478766 & 0.997681351 \\ \hline
Precision       & 0.880131363   & 0.838567839                & 0.990819672      & 0.85584563  & 0.99867374  \\ \hline
Recall          & 0.709463931   & 0.883520847                & 1                & 0.99801456  & 0.996690933 \\ \hline
timePerformance & 26782         & 38772                      & 3899             & 2247        & 33966       \\ \hline
\end{tabular}%
}
\caption{SPIMBENCH MAINBOX 2020}
\label{tab:SPIMBENCH_MAINBOX_2020}
\end{table}

\begin{table}[ht!]
\resizebox{\textwidth}{!}{%
\begin{tabular}{|l|l|l|l|l|l|l|l|l|l|l|l|l|l|l|l|l|l|l|}
\hline
\multicolumn{1}{|c|}{\textbf{}}       & \multicolumn{1}{c|}{\textbf{}}     & \multicolumn{4}{c|}{\textbf{class}}                                                                                                                       & \multicolumn{4}{c|}{\textbf{property}}                                                                                                             & \multicolumn{4}{c|}{\textbf{instance}}                                                                                                             & \multicolumn{4}{c|}{\textbf{overall}}                                                                                                              &                                    \\ \hline
\multicolumn{1}{|c|}{\textbf{System}} & \multicolumn{1}{c|}{\textbf{Time}} & \multicolumn{1}{c|}{\textbf{\#testcases}} & \multicolumn{1}{c|}{\textbf{Size}} & \multicolumn{1}{c|}{\textbf{Prec.}} & \multicolumn{1}{c|}{\textbf{F-m.}} & \multicolumn{1}{c|}{\textbf{Rec.}} & \multicolumn{1}{c|}{\textbf{Size}} & \multicolumn{1}{c|}{\textbf{Prec.}} & \multicolumn{1}{c|}{\textbf{F-m.}} & \multicolumn{1}{c|}{\textbf{Rec.}} & \multicolumn{1}{c|}{\textbf{Size}} & \multicolumn{1}{c|}{\textbf{Prec.}} & \multicolumn{1}{c|}{\textbf{F-m.}} & \multicolumn{1}{c|}{\textbf{Rec.}} & \multicolumn{1}{c|}{\textbf{Size}} & \multicolumn{1}{c|}{\textbf{Prec.}} & \multicolumn{1}{c|}{\textbf{F-m.}} & \multicolumn{1}{c|}{\textbf{Rec.}} \\ \hline
ALOD2Vec                              & 0:13:24                            & 5                                         & 20.0                               & 1.00 (1.00)                         & 0.80 (0.80)                        & 0.67 (0.67)                        & 76.8                               & 0.94 (0.94)                         & 0.95 (0.95)                        & 0.97 (0.97)                        & 4893.8                             & 0.91 (0.91)                         & 0.87 (0.87)                        & 0.83 (0.83)                        & 4990.6                             & 0.91 (0.91)                         & 0.87 (0.87)                        & 0.83 (0.83)                        \\ \hline
AML                                   & 0:50:55                            & 5                                         & 23.6                               & 0.98 (0.98)                         & 0.89 (0.89)                        & 0.81 (0.81)                        & 48.4                               & 0.92 (0.92)                         & 0.70 (0.70)                        & 0.57 (0.57)                        & 6802.8                             & 0.90 (0.90)                         & 0.85 (0.85)                        & 0.80 (0.80)                        & 6874.8                             & 0.90 (0.90)                         & 0.85 (0.85)                        & 0.80 (0.80)                        \\ \hline
ATBox                                 & 0:16:22                            & 5                                         & 25.6                               & 0.97 (0.97)                         & 0.87 (0.87)                        & 0.79 (0.79)                        & 78.8                               & 0.97 (0.97)                         & 0.96 (0.96)                        & 0.95 (0.95)                        & 4858.8                             & 0.89 (0.89)                         & 0.84 (0.84)                        & 0.80 (0.80)                        & 4963.2                             & 0.89 (0.89)                         & 0.85 (0.85)                        & 0.81 (0.81)                        \\ \hline
baselineAltLabel                      & 0:10:57                            & 5                                         & 16.4                               & 1.00 (1.00)                         & 0.74 (0.74)                        & 0.59 (0.59)                        & 47.8                               & 0.99 (0.99)                         & 0.79 (0.79)                        & 0.66 (0.66)                        & 4674.8                             & 0.89 (0.89)                         & 0.84 (0.84)                        & 0.80 (0.80)                        & 4739.0                             & 0.89 (0.89)                         & 0.84 (0.84)                        & 0.80 (0.80)                        \\ \hline
baselineLabel                         & 0:10:44                            & 5                                         & 16.4                               & 1.00 (1.00)                         & 0.74 (0.74)                        & 0.59 (0.59)                        & 47.8                               & 0.99 (0.99)                         & 0.79 (0.79)                        & 0.66 (0.66)                        & 3641.8                             & 0.95 (0.95)                         & 0.81 (0.81)                        & 0.71 (0.71)                        & 3706.0                             & 0.95 (0.95)                         & 0.81 (0.81)                        & 0.71 (0.71)                        \\ \hline
DESKMatcher                           & 0:13:54                            & 5                                         & 91.4                               & 0.76 (0.76)                         & 0.71 (0.71)                        & 0.66 (0.66)                        & 0.0                                & 0.00 (0.00)                         & 0.00 (0.00)                        & 0.00 (0.00)                        & 3820.6                             & 0.94 (0.94)                         & 0.82 (0.82)                        & 0.74 (0.74)                        & 3912.0                             & 0.93 (0.93)                         & 0.81 (0.81)                        & 0.72 (0.72)                        \\ \hline
LogMap                                & 2:55:14                            & 5                                         & 24.0                               & 0.95 (0.95)                         & 0.84 (0.84)                        & 0.76 (0.76)                        & 0.0                                & 0.00 (0.00)                         & 0.00 (0.00)                        & 0.00 (0.00)                        & 29190.4                            & 0.40 (0.40)                         & 0.54 (0.54)                        & 0.86 (0.86)                        & 29214.4                            & 0.40 (0.40)                         & 0.54 (0.54)                        & 0.84 (0.84)                        \\ \hline
LogMapBio                             & 4:35:29                            & 5                                         & 24.0                               & 0.95 (0.95)                         & 0.84 (0.84)                        & 0.76 (0.76)                        & 0.0                                & 0.00 (0.00)                         & 0.00 (0.00)                        & 0.00 (0.00)                        & 0.0                                & 0.00 (0.00)                         & 0.00 (0.00)                        & 0.00 (0.00)                        & 24.0                               & 0.95 (0.95)                         & 0.01 (0.01)                        & 0.00 (0.00)                        \\ \hline
LogMapIM                              & 2:49:34                            & 5                                         & 0.0                                & 0.00 (0.00)                         & 0.00 (0.00)                        & 0.00 (0.00)                        & 0.0                                & 0.00 (0.00)                         & 0.00 (0.00)                        & 0.00 (0.00)                        & 29190.4                            & 0.40 (0.40)                         & 0.54 (0.54)                        & 0.86 (0.86)                        & 29190.4                            & 0.40 (0.40)                         & 0.54 (0.54)                        & 0.84 (0.84)                        \\ \hline
LogMapKG                              & 2:47:51                            & 5                                         & 24.0                               & 0.95 (0.95)                         & 0.84 (0.84)                        & 0.76 (0.76)                        & 0.0                                & 0.00 (0.00)                         & 0.00 (0.00)                        & 0.00 (0.00)                        & 29190.4                            & 0.40 (0.40)                         & 0.54 (0.54)                        & 0.86 (0.86)                        & 29214.4                            & 0.40 (0.40)                         & 0.54 (0.54)                        & 0.84 (0.84)                        \\ \hline
LogMapLt                              & 0:07:19                            & 4                                         & 23.0                               & 0.80 (1.00)                         & 0.56 (0.70)                        & 0.43 (0.54)                        & 0.0                                & 0.00 (0.00)                         & 0.00 (0.00)                        & 0.00 (0.00)                        & 6653.8                             & 0.73 (0.91)                         & 0.67 (0.84)                        & 0.62 (0.78)                        & 6676.8                             & 0.73 (0.92)                         & 0.66 (0.83)                        & 0.61 (0.76)                        \\ \hline
Wiktionary                            & 0:30:12                            & 5                                         & 22.4                               & 1.00 (1.00)                         & 0.80 (0.80)                        & 0.67 (0.67)                        & 80.0                               & 0.94 (0.94)                         & 0.95 (0.95)                        & 0.97 (0.97)                        & 4893.8                             & 0.91 (0.91)                         & 0.87 (0.87)                        & 0.83 (0.83)                        & 4996.2                             & 0.91 (0.91)                         & 0.87 (0.87)                        & 0.83 (0.83)                        \\ \hline
\end{tabular}%
}
\caption{Knowledge Graph 2020 Overview}
\label{tab:Knowledge_Graph_2020_Overview}
\end{table}

\begin{table}[ht!]
\resizebox{\textwidth}{!}{%
\begin{tabular}{|l|l|l|l|l|l|l|l|l|l|l|l|l|l|l|l|l|l|l|l|l|}
\hline
\multicolumn{4}{|c|}{\textbf{marvelcinematicuniverse-marvel}}                                                                                   & \multicolumn{4}{c|}{\textbf{memoryalpha-memorybeta}}                                                                                               & \multicolumn{4}{c|}{\textbf{memoryalpha-stexpanded}}                                                                                               & \multicolumn{4}{c|}{\textbf{starwars-swg}}                                                                                                         & \multicolumn{4}{c|}{\textbf{starwars-swtor}}                                                                                                       &                                    \\ \hline
\multicolumn{1}{|c|}{\textbf{}} & \multicolumn{1}{c|}{\textbf{Size}} & \multicolumn{1}{c|}{\textbf{Prec.}} & \multicolumn{1}{c|}{\textbf{F-m.}} & \multicolumn{1}{c|}{\textbf{Rec.}} & \multicolumn{1}{c|}{\textbf{Size}} & \multicolumn{1}{c|}{\textbf{Prec.}} & \multicolumn{1}{c|}{\textbf{F-m.}} & \multicolumn{1}{c|}{\textbf{Rec.}} & \multicolumn{1}{c|}{\textbf{Size}} & \multicolumn{1}{c|}{\textbf{Prec.}} & \multicolumn{1}{c|}{\textbf{F-m.}} & \multicolumn{1}{c|}{\textbf{Rec.}} & \multicolumn{1}{c|}{\textbf{Size}} & \multicolumn{1}{c|}{\textbf{Prec.}} & \multicolumn{1}{c|}{\textbf{F-m.}} & \multicolumn{1}{c|}{\textbf{Rec.}} & \multicolumn{1}{c|}{\textbf{Size}} & \multicolumn{1}{c|}{\textbf{Prec.}} & \multicolumn{1}{c|}{\textbf{F-m.}} & \multicolumn{1}{c|}{\textbf{Rec.}} \\ \hline
ALOD2Vec                        & 3065                               & 0.86                                & 0.76                               & 0.68                               & 13325                              & 0.92                                & 0.91                               & 0.90                               & 3285                               & 0.92                                & 0.93                               & 0.93                               & 2118                               & 0.92                                & 0.82                               & 0.75                               & 2676                               & 0.92                                & 0.92                               & 0.91                               \\ \hline
AML                             & 4670                               & 0.85                                & 0.68                               & 0.56                               & 18319                              & 0.91                                & 0.89                               & 0.87                               & 3699                               & 0.93                                & 0.93                               & 0.93                               & 3491                               & 0.90                                & 0.81                               & 0.75                               & 3835                               & 0.93                                & 0.92                               & 0.90                               \\ \hline
ATBox                           & 3483                               & 0.66                                & 0.58                               & 0.52                               & 12860                              & 0.96                                & 0.93                               & 0.91                               & 3162                               & 0.96                                & 0.94                               & 0.92                               & 2170                               & 0.93                                & 0.83                               & 0.76                               & 2619                               & 0.94                                & 0.92                               & 0.91                               \\ \hline
baselineAltLabel                & 2559                               & 0.86                                & 0.76                               & 0.68                               & 13454                              & 0.88                                & 0.89                               & 0.89                               & 3165                               & 0.88                                & 0.90                               & 0.93                               & 1661                               & 0.92                                & 0.74                               & 0.62                               & 2535                               & 0.92                                & 0.91                               & 0.90                               \\ \hline
baselineLabel                   & 1864                               & 0.90                                & 0.69                               & 0.56                               & 10492                              & 0.95                                & 0.85                               & 0.77                               & 2517                               & 0.98                                & 0.91                               & 0.84                               & 1194                               & 0.95                                & 0.67                               & 0.52                               & 2142                               & 0.95                                & 0.89                               & 0.84                               \\ \hline
DESKMatcher                     & 1898                               & 0.89                                & 0.69                               & 0.56                               & 10831                              & 0.94                                & 0.85                               & 0.78                               & 2557                               & 0.98                                & 0.91                               & 0.84                               & 1522                               & 0.93                                & 0.76                               & 0.64                               & 2295                               & 0.94                                & 0.90                               & 0.86                               \\ \hline
LogMap                          & 41070                              & 0.24                                & 0.35                               & 0.71                               & 54499                              & 0.43                                & 0.58                               & 0.87                               & 15529                              & 0.47                                & 0.63                               & 0.94                               & 15323                              & 0.58                                & 0.68                               & 0.82                               & 19531                              & 0.27                                & 0.42                               & 0.96                               \\ \hline
LogMapBio                       & 0                                  & 0.00                                & 0.00                               & 0.00                               & 0                                  & 0.00                                & 0.00                               & 0.00                               & 0                                  & 0.00                                & 0.00                               & 0.00                               & 0                                  & 0.00                                & 0.00                               & 0.00                               & 0                                  & 0.00                                & 0.00                               & 0.00                               \\ \hline
LogMapIM                        & 41070                              & 0.24                                & 0.35                               & 0.71                               & 54499                              & 0.43                                & 0.58                               & 0.87                               & 15529                              & 0.47                                & 0.63                               & 0.94                               & 15323                              & 0.58                                & 0.68                               & 0.82                               & 19531                              & 0.27                                & 0.42                               & 0.96                               \\ \hline
LogMapKG                        & 41070                              & 0.24                                & 0.35                               & 0.71                               & 54499                              & 0.43                                & 0.58                               & 0.87                               & 15529                              & 0.47                                & 0.63                               & 0.94                               & 15323                              & 0.58                                & 0.68                               & 0.82                               & 19531                              & 0.27                                & 0.42                               & 0.96                               \\ \hline
LogMapLt                        & 0                                  & 0.00                                & 0.00                               & 0.00                               & 16665                              & 0.90                                & 0.83                               & 0.77                               & 3550                               & 0.94                                & 0.89                               & 0.84                               & 2795                               & 0.91                                & 0.75                               & 0.63                               & 3605                               & 0.91                                & 0.89                               & 0.87                               \\ \hline
Wiktionary                      & 3065                               & 0.86                                & 0.76                               & 0.68                               & 13327                              & 0.92                                & 0.91                               & 0.90                               & 3283                               & 0.92                                & 0.92                               & 0.93                               & 2118                               & 0.92                                & 0.82                               & 0.75                               & 2676                               & 0.92                                & 0.92                               & 0.91                               \\ \hline
\end{tabular}%
}
\caption{Knowledge Graph 2020 Instance Results}
\label{tab:Knowledge_Graph_2020_Instance_Results}
\end{table}

\section{System Development}

Various tools, evaluation frameworks and APIs are provided by the ontology matching industry, and can be utilised in the development of matching systems. An introduction to the ones related to this project is given below.

\subsection{OWL Tools and Applications}

A number of tools are available for ontology engineering tasks such as development and maintenance\footnote{\url{https://www.w3.org/wiki/Ontology_editors}}. Protégé\footnote{\url{https://protege.stanford.edu}} \cite{DBLP:journals/aimatters/Musen15} is an open-source ontology editor with a friendly user interface and lots of plug-ins such as OWL/RDF parsers or description logic reasoners. Besides basic editing over ontologies, functionalities such as converting axioms and reasoning about inconsistencies can be achieved. It has been used for manipulating the toy datasets and visualising the class hierarchies of OWL 2 ontologies, and will be used for general purpose ontology editing in the later stages.
\\\\
For the visualisation of alignments, tools such as HOMER and AlViz \cite{DBLP:conf/iv/LanzenbergerS06} exist. Although visualising the alignments is not within the objectives of the project, it is beneficial to use such tools to check the results of smaller-scaled matching tasks. Tools for ontology versioning, merging and modularisation also exist \cite{DBLP:conf/ecai/Jimenez-RuizGZH12}.

\subsection{APIs}

The OWL API\footnote{\url{https://github.com/owlcs/owlapi}} can be used for creating, manipulating and serialising OWL ontologies, and its latest version is OWL 2 compatible. Core functionalities implemented in the OWL API include: parsing, rendering and writing OWL ontologies in various syntaces, extracting and organising axioms, interfacing with description logic reasoners, etc. It is also a hard dependency for many tools and other APIs. As the implementation will target the matching of OWL 2 ontologies only, the OWL API is the absolute necessity throughout the implementation stages.
\\\\
For the most powerful description logic reasoners such as HermiT \cite{DBLP:journals/jar/GlimmHMSW14} and FaCT++ \cite{DBLP:conf/cade/TsarkovH06}, they can not only be used as plug-ins for Protégé, but also be used as Java APIs. This means that specific functionalities can be invoked from the code instead of having to run the whole reasoner. These reasoner APIs are crucial in the development stage of reasoning-based techniques, and also in the evaluation stage for reasoning about the matched instances after the majority of the implementation tasks have finished. It should be noted that becides the general-purpose reasoners mentioned above, profile-specific reasoners are also available. For example, the ELK \cite{DBLP:journals/jar/KazakovKS14} reasoner targets the OWL 2 EL profile, which has very restricted expressiveness compared to the complete OWL 2. However, because of the huge reduction in the complexity of axioms to be considered, ELK performs reasoning on OWL 2 EL ontologies much faster than the general-purpose reasoners. Lists of description logic reasoners can be found on the official websites of W3C and the University of Manchester\footnote{\url{https://www.w3.org/2001/sw/wiki/OWL/Implementations}}\footnote{\url{http://owl.cs.manchester.ac.uk/tools/list-of-reasoners}}.
\\\\
The Alignment API\footnote{\url{http://alignapi.gforge.inria.fr}} can be used for generating and sharing ontology alignments that are compatible with the evaluation frameworks. If the implementation needs to be wrapped in one of the evaluation frameworks to perform local tests, it is necessary to use the Alignment API so that comparable matching results can be generated.
\\\\
Since the ultimate aim of this project is to support the tasks of data integration, queires will be performed over the respective ontologies with reference to the generated alignments. This can be done with the help of SPARQL-Generate API\footnote{\url{https://github.com/sparql-generate}}, which provides a standard mechanism to search for semantic contents.

\subsection{Evaluation Frameworks}

The Ontology Alignment Evaluation Initiative (OAEI) provides two evaluation frameworks, SEALS\footnote{\url{http://oaei.ontologymatching.org/2020/seals/index.html}} and HOBBIT\footnote{\url{http://oaei.ontologymatching.org/2020/hobbit/index.html}}, that can be adopted for the running of different benchmark tracks. A total of 12 evaluation tracks are provided in the two benchmark frameworks: Anatomy (SEALS), Conference (SEALS), Multifarm (SEALS), Complex (SEALS), Interactive matching (SEALS), Largebio (SEALS, HOBBIT), Phenotype (SEALS), Biodiversity and Ecology (SEALS), SPIMBENCH (HOBBIT), Link Discovery (HOBBIT), GeoLink Cruise (SEALS) and Knowledge Graph (SEALS, HOBBIT), whose detailed descriptions can be found on the OAEI Workshop 2020 website\footnote{\url{http://oaei.ontologymatching.org/2020/}}. While most of them focus on the evaluation of schema matching, three of them, namely SPIMBENCH, Link Discovery, and Knowledge Graph, are suitable for the evaluation of instance matching.
\\\\
SPIMBENCH seeks to determine whether two instances describe the same Creative Work, where value-based, structure-based, and semantics-aware transformations are made to the original dataset to produce the new dataset to be matched with. This is ideal for the instance matching evaluation of this project. Link Discovery focuses more on spacial data, and Knowledge Graph aims at matching both schema and instances. These two can be used to complement the evaluation result to be more general. As all three of these are provided within the HOBBIT framework, the implementation will be wrapped in HOBBIT using the HOBBIT Java SDK\footnote{\url{https://github.com/hobbit-project/java-sdk}}.


% \chapter{Design}

Containing a comprehensive description of the design chosen, how it addresses the problem, and why it is designed the way it is.

\chapter{Evaluation}

Explaining how your software was tested (using different datasets or in different environments), statistical evaluation of performance, results of user evaluation questionnaires, etc.

\chapter{Implementation}

Containing a comprehensive description of the implementation of your software, including the language(s) and platform chosen, problems encountered, any changes made to the design as a result of the implementation, etc.


% \chapter{Reflection}

This chapter illustrates various aspects reside in the management of the project, including a revision of the original workplan, a description of the working methodology, resource management details and contingency measures. A personal reflection to the project is also presented. 

\section{Project Management}
% Covering the tasks as a part of your work plan and progress as well as how time and resources are managed.

\subsection{Revision of Workplan}
% Revision of workplan.
Figure \ref{fig:Gantt_old} shows the proposed workplan of the project designed at the beginning stage. Generally speaking, the progress of the project has been satisfying, as all the phases involving literature review and systematic learning have been completed on time. One task was shelved and left behind schedule, however, namely the delivery of an ontology matching framework, which was supposed to come along with the interim report. There are two primary contributing factors to the making of this decision. Firstly, the workload in studying existing ontology matching techniques was underestimated, making the expectation of designing the framework simultaneously with the learning process unrealistic. Secondly, time allocation was inevitably tilted to the investigation of instance matching strategies, as they are inseparable from the general ontology matching framework. This was considered better for the selection of suitable framework components after some investigation into ontology matching, and was hence given higher priority.
\\\\
In order to reflect the above changes, a revised workplan was designed after reaching the Interim Report milestone, and is presented in figure \ref{fig:Gantt_new}. The new workplan demonstrates how the previous stages had been completed, as well as how the second half of the project was designed to progress. Due to the effect of autumn semester exam preparation, implementation tasks before the exams were only preliminary, while the major development tasks were put in the winter vacation. Since coding and debugging can take a large amount of time, this decision was proved to be sensible. The adjustment of ontology matching algorithm was redesigned to be side by side with testing and evaluation, with the validation of design carried all along the way. This was to target the unexpected obstacles and potential risks such as increased academic workload during the spring semester, in which case the adjustment phase can shrink to buffer the impact on the overall workplan. This was proved to be extremely effective, as unforeseen difficulties within the project implementation stage could not have been countered without the degree of flexibility in the timeframe. The workload of writing the final dissertation was spread over a longer timespan, as it was discovered that writing usually takes longer than expected. Although time was still scheduled very tightly at the final stage of the project because of the difficulty of implementation, everything was able to be delivered on time thanks to following the revised workplan.

\begin{figure}[ht]
\begin{subfigure}[ht]{0.5\textwidth}
\includegraphics[width=\textwidth]{img/Gantt_old.pdf}
\caption{Original workplan}
\label{fig:Gantt_old}
\end{subfigure}
\begin{subfigure}[ht]{0.5\textwidth}
\includegraphics[width=\textwidth]{img/Gantt_new.pdf}
\caption{Revised workplan}
\label{fig:Gantt_new}
\end{subfigure}
\caption{Revision of workplan}
\label{fig:Gantt}
\end{figure}

\subsection{Working Methodology}
% Scrumban methodology.
While the workplan is designed with respect to the Waterfall methodology \cite{balaji2012waterfall} for clear validation of progress against the milestones, a hybrid combination of the Scrum and Kanban methodologies, namely Scrumban \cite{DBLP:conf/ispw/NikitinaKS12}, was adopted and followed as the working methodology since the beginning. The Scrum aspect splits the workload into small, easily achievable goals which could be reviewed regularly, while a Kanban board enables clear monitoring of the process, with tasks being assigned in dynamic ``lanes'' (see figure \ref{fig:Kanban}\footnote{\url{https://github.com/Jizhou-Che/Dissertation/}}). Fusing the two together exploits the advantages of both: the workflow can adapt to changes quickly, and since all tasks are visualised, the efforts can be well balanced and coordinated. The sprint interval for Scrumban was set to 1 week, which is the interval between every consecutive meetings with my supervisor. This helped tremendously in pushing everything forward, so that the project can progress at a good pace despite the intense curricular work and my lack of experience in carrying out a large project like this individually.

\begin{figure}[ht]
\includegraphics[width=\textwidth]{img/Kanban.png}
\caption{Kanban Board}
\label{fig:Kanban}
\end{figure}

\subsection{Resource Management}
% Resource management.
All resources related to the project, such as meeting minutes, reading notes and source code have been managed with the Git version control system and consistently pushed to GitHub. They can be found at \url{https://github.com/Jizhou-Che/Dissertation}. This has proven to be a good practice, in that the evolution of the project is documented within the accumulation and development of resources. The project have been adhering to this methodology throughout.

\subsection{Contingency Measures}
% Risks that may arise in the revised workplan, and how they can be mitigated and managed.
Admittedly, I did not put too much thought into contingency measures at the start of the project. However, I have learned its necessity though my experience in managing this long-term project. Based on the issues I have encountered in the first part of the project as well as my personal observations, the most significant risks that can arise in the process are pointed out below, as well as the proposed strategies to mitigate their effects.
\\\\
Underestimating the steepness of learning curves or the difficulty of implementation can disrupt the regular time allocation, and therefore shake the entire workplan. Apart from foreseeing the issue and allocate adequate amount of time in advance, time can be borrowed from flexible tasks that are less affected by a shrink of timespan. While this provides a temporary solution that mitigates the impact to the workplan, ultimately extra time should be dedicated to the project as compensation.
\\\\
The workloads of other modules is another serious issue, especially when their deadlines are close to the project milestones. This can not only disrupt the time management, but also increase the psychological pressure. Low moods and anxiety can lead to a drop in the productivity as well as the quality of work produced. Having tried the relax-then-focus approach, that was not suitable for me as the guilty only increased. Instead, trying to finish the work together with a friend has proved to be more effective.
\\\\
Offline meetings can sometimes be inappropriate due to wellbeing or other issues. In such cases online meetings via Zoom were organised, and have proven equal effectiveness.


\section{Reflections}
% A personal reflection on the plan and your experience of the project (a critical appraisal of how the project went).
I have faced challenges in various aspects of the project. Academically, the steep learning curve of background knowledge made the workload that was already intense even heavier, and the locating of quality resources for knowledge acquisition required professional judgment and guidance especially at the beginning. My supervisor helped me a lot in such aspects, so I could dive quickly into solid work. For example, after given a brief introduction to description logic as the foundation of OWL ontologies, I picked some books and online learning materials for self-study. My supervisor helped assessing the quality of my selections, and pointed out where I should investigate first to get started, which was extremely helpful to me. My supervisor also suggested excellent ideas when I needed to consider different approaches. For example, when producing an initial set of rules used for reasoning, my supervisor suggested adding assumptions such as disjoint siblings to the set. This has proved to increase the number of correct mappings produced especially for ontologies with poorly defined schema-level information. At times when I got stuck for lacking knowledge or data, my supervisor directed me to prestigious professors, papers and websites so I could acquire just the resources I needed. This equipped me with solid understanding about the core subjects I had been dealing with throughout the project. I also got a lot of help regarding academic writing from my supervisor, in the form of either oral instructions or written feedback. These served as a golden standard whenever I needed to draft something professional.
\\\\
Regarding the implementation steps, when exploring the prominent ontology matching systems at the beginning, it seemed intimidating that even a lite version of a matching system is constructed from so many components, with thousands of lines of code to investigate within each of them. Although being comfortable with Java programming, I did not know where to start for quite some time. It was after I have accumulated sufficient amount of knowledge by reading the relevant literatures and supporting documents that the meaning of code segments became clearer to me. Also, testing the entire system gives only the matching result, which did not contribute to the understanding of the programme itself. Therefore, I started by extracting individual modules and classes and tested them with intermediate results generated by debugging tools. This ultimately enabled me to build up a solid understanding of the logic and semantics of the programmes, so I could know where my own implementation should start with, as well as how some of the existing modules can be adapted to my implementation framework. My supervisor inspired me with implementation ideas as well. For example, when I got stuck at parsing the reference alignment from RDF to TXT format, my supervisor shared some existing code pieces with me, and by investigating them I discovered that it is possible to render all acceptable formats into an internal data structure, which simplified my implementation greatly.
\\\\
Considering project management, time allocation was the largest difficulty I encountered, as it was hard to balance the project with other curricular subjects, and sometimes my judgment of the amount of time required for a task was inadequate. Sometimes I feel like I could have used some of my time more wisely, for instance, investigating matching systems based on fuzzy logic did not really help with my implementation, but there was no telling that without having done the work. Therefore, I consider this as a necessary stage that must be gone through to toughen my skills and enrich my experience, and I feel I have made advancements in my time management abilities at the end of the project. Managing the resources is what I did best. By preparing for the meetings and keeping the minutes well-organised at all time, I hardly had any trouble locating any information that was previously discussed in the meetings. Academic resources such as books, research papers and evaluation websites were indexed separately so that they are always easy to access. My own writing and implementation have been managed using version control since the beginning, so I never suffered from lost of data under any circumstances.
\\\\
I feel truly fulfilled at the end of the project, having learned a lot of fundamental theories about logical reasoning and knowledge representation and processing. These are subjects I wished to learn before doing the project, and I am proud to be able to achieve my goals in the end. Apart from that, I find myself becoming a competent user of OWL ontology language, semantic web technology, the related tools, and of course, Java programming. These are solid skills that could not be acquired without practicing with them in a large project. Finally, I am no longer terrified by large projects development and management, as even the most complex project can be done though cumulative effort, which is exactly what I have done through this project. I will certainly benefit from all of these invaluable experience in my future academic works.


\pagestyle{plain}
\pagenumbering{arabic}

\singlespacing

\section{Background and Motivation}

% (1 page) Briefly describe the background to the project, importance/need of the area, and motivation for carrying out the proposed work.
% (Explaining the problem being solved.)

The ability of using data from multiple sources simultaneously has always been demanded in both research and public areas. For example, translational research data from multiple biomedical domains can be integrated to track and analyze human-centered records \cite{wang2009translational}, and data exported from each peer in its own schema need to be integrated in a P2P system \cite{calvanese2003semantic}. These applications are benefited from the integrated datasets, in that the semantic interconnections between information from multiple sources are identified, providing substantial new possibilities in inter-operating and interpreting the data.
\\\\
Data integration in the context of the traditional web has been challenging. Firstly, the accessibility and usability of data are unpromising, as resources can be stored in heterogeneous formats that are not easily readable to the computers, such as Excel or scanned PDF. In addition, without a formal representation of the semantics of the knowledge within the data, the computers can only interpret the datasets at a very basic level \cite{shadbolt2006semantic}. Further more, the matching process needs to be carried out even if the datasets come from the same domain of interest, due to the lack of a uniform data modeling principle \cite{shvaiko2007ontology}.
\\\\
The idea of linked data and semantic web provides a fundamental framework for addressing these issues. Resources are uniquely identified by the Uniform Resource Identifiers (URIs), meaning that references or links to resources can be specifically established; The Resource Description Framework (RDF) defines the universal structure and captures the semantics of the data, to provide the power of interpreting and querying the datasets; The Web Ontology Language (OWL) enables efficient representation of ontologies, promoting a shared understanding of a knowledge domain by explicitly and formally specifying the objects and relations \cite{shadbolt2006semantic}. Rules can be defined with rule markup languages (RuleML) and also expressed in SPARQL queries \cite{polleres2007sparql}.
\\\\
Based on these concepts, the idea of ontology matching has evolved to become a solution to data integration. Information coming from multiple local sources are integrated without producing a newly merged dataset \cite{wache2001ontology}. In order to do this, a common ontology need to be built as an inter-connection between the ontologies of the domains where the data are sourced from \cite{shvaiko2007ontology}. This includes the matching of TBoxes, ABoxes and sometimes RBoxes, which are fundamental concepts of description logic \cite{baader2017introduction}. A query posed over the common ontology is decomposed into multiple queries over the respective datasets. This approach addresses the semantic heterogeneity problem in data integration to some extent \cite{vaishnavi2005semantic}.
\\\\
A number of previous works have been done in the research of ontology matching, suggesting approaches that fall into various categories \cite{shvaiko2007ontology}. However, the ones that target the application in data integration are not clearly recognised. An automated framework for general-purpose data integration is also absent, as current applications usually define the correspondences manually with the knowledge of domain experts for guaranteed correctness and completeness \cite{cruz2012interactive}. It is hence sensible to implement one that embeds the properly adapted strategies. The challenges reside in selecting the characteristics from ABoxes and TBoxes for matching, where a feature may introduce positive and negative effects on the result in different scenarios \cite{shvaiko2011ontology}. This project looks specifically into the matching of ABoxes that suits the needs of data integration.
% \\\\
% The traditional solutions to dataset integration typically relies on matching the semantically related entities at the schema level [Ontology matching, 8; Batini et al. 1986; Sheth and Larson 1990; Spaccapietra and Parent 1991; Parent and Spaccapietra 1998] or the catalog level [Bouquet et al. 2003b, Bernstein and Rahm 2000]. The drawback of such approaches is obvious: updates on the individual datasets are not reflected in the merged dataset without extra work. Further more, the matching process needs to be carried out even if the datasets come from the same domain of interest, due to the lack of a uniform data modeling principle [ontology matching, 8].

% -----------------------------------------------------------------------------------------------------------------------------------------------------------

% 1. Data integration: demand, application, importance, benefits

% 2. Data integration challenges: semantic heterogeneity problem [], data fetching usability, data interpretation semantic analysis ---

% 3. Linked data and semantic web: address the issue, identifying resources (URI), modeling data graphically (RDF), give semantics to domain terminologies (ontology) WHY THIS IS THE DIRECTION TO GO

% 4. Ontology matching: the approach to data integration. PREVIOUS WORK

% 5. current ontology matching strategies analysis: all manually, no automated implementation for data integration: the need to implement one.

% 6. Potential challenges.

% -----------------------------------------------------------------------------------------------------------------------------------------------------------

% The need of using information simultaneously from multiple sources has been dramatically increasing in the recent researches [? citation] \cite{placeholder}. For example, datasets of different urban infrastructure assets can be integrated to produce a decision support system for city management [heshan], and --- []. Unfortunately, resources are usually constructed and maintained independently in heterogeneous formats [? citation] \cite{placeholder}. The traditional approach of making use of the interdependencies between datasets relies on matching the semantically related entities at the schema level [Ontology matching, 8; Batini et al. 1986; Sheth and Larson 1990; Spaccapietra and Parent 1991; Parent and Spaccapietra 1998] \cite{placeholder}. An obvious drawback of this approach is that updates performed on the individual datasets are not reflected in the merged dataset without extra work. Further more, the matching process needs to be carried out even if the datasets come from the same domain of interest, due to the lack of a uniform modeling principle [ontology matching, 8]\cite{placeholder}.
% \\\\
% Another challenge resides in that the usability of data in the traditional web is usually questionable: datasets can be represented in formats such as CSV, Excel, HTML, PDF table or even scanned pictures [Ontology matching, 11, linked data]. Computers play a very limited role in interpreting such media, doing solely words indexing and document serving, while the intelligent works such as selecting, combining and aggregating the data are performed by human[citation: A Semantic Web Primer, series foreword, xvi].
% \\\\
% Semantic web concepts and techniques have been established to conquer the above issues. The fundamental goal of the semantic web is to make data on the web more readable and understandable by the machines. First of all, resources on the web are uniquely identified by the Uniform Resource Identifiers (URIs). The URIs are used for distinguishing resource in the Resource Description Framework (RDF), the predominant method of conceptually describing and modeling information [citation: ?]. RDF serves as a standard and universal principle of data modeling, so that datasets from distributed sources that are referencing the same resources in the same knowledge domain can be combined together without confusion. Further more, data expressed in RDF are made available in structured and standarised formats that can be easily fetched by computers using a query language such as SPARQL [? citation], increasing the usability of data dramatically.
% \\\\
% Based on the concept of the semantic web, the research of data integration propose the approach where integration of information coming from multiple local sources is performed without first loading the data into a central warehouse [Chawathe et al. 1994; Wache et al. 2001; Draper et al. 2001; --Halevy et al. 2005--; Seligman et al. 2010; Doan et al. 2012;]\cite{placeholder}. This is achieved with the use of ontologies [? citation]. Ontologies provide a shared understanding of a domain by explicitly and formally specifying the concepts in the domain [? citation]. OWL 2, being the current standard of ontology languages, is based on RDF. This structure makes the inter-operation across multiple local sources possible. First, a common ontology need to be built as an inter-connection between the ontologies of the domains where the data are sourced from [Ontology matching, 10]. A query posed over the common ontology are decomposed into multiple queries over the domain ontologies of the respective datasets. This process is known as ontology matching [citation].
% \\\\
% Although various automatic methods for matching data and ontologies have been proposed \cite{placeholder}, correspondences are usually manually defined by domain experts for guaranteed correctness [citation]. This is because that the correctness of matching results cannot be guaranteed without domain expert knowledge. The challenge is arisen from the strategy used to match ontologies. For example, terminologies with similar names can have completely different semantics or meanings, and entries in different datasets can represent the same terminology but have different data-types or units or measuring standards. Although methods exist for quantifying the level of certainty [citation], they do not target for the application of data integration, where the level of correctness can be potentially increased. This project looks specifically into the matching process of ABoxes to target this issue.
% \\\\
% Description logics (DLs) are a family of knowledge representation languages that can be used to represent knowledge of an application domain in a structured and well-understood way [Citation: An Introduction to Description Logic, 1] \cite{placeholder}. Recently, DLs have played a central role in the semantic web [Hor08], having been adopted as the basis for ontology languages [HPSvH03]. They typically separate domain knowledge into two components: the terminological part TBox and the assertional part ABox, where the combination of a TBox and an ABox is called a knowledge base (KB) [Citation: An Introduction to Description Logic, 1] \cite{placeholder}. A crucial feature of DLs is that terminological and assertional statements have a formal logic-based semantics, meaning that entailments of such statements in a knowledge base can be determined through automated reasoning [Citation: An Introduction to Description Logic, 1] \cite{placeholder}.


\section{Aims and Objectives}

% (0.5 page) The aim is usually a single sentence describing at a high-level what the point of the project is and what will be achieved.
% The objectives are sub-components of the general aim, detailing the individual aspects which need to be achieved in order to deliver the aim(s).

The aim of this project is to provide an automated solution for data integration, and to deliver an application as a tangible result. Specifically, this project attempts to come up with an ABox matching strategy suitable for the needs of data integration. In order to achieve this, the following objectives need to be reached:
\begin{enumerate}[itemsep=1.1em]
	\item Investigate and adopt existing algorithms and strategies to automate the production of a common ontology.
	\item Explore and manipulate the strategies in the automation of ABox matching, to design one specifically suitable for data integration.
	\item Investigate the semantic propagation and preservation in query answering.
	\item Deliver a tangible implementation of the system.
	\item Testing, evaluating, and potentially reasoning about the implementation.
\end{enumerate}
It should be noted that it is not the target of the project to achieve complete correctness, which is considered impractical without the aid of domain expert knowledge. Although, the level of precision and completeness is considered as one of the most central evaluation criterion, as well as the underlying reasonability and decidability of the logic.


\section{Work Plan}

% (1 page) Describing the tasks to be carried out.
% The time plan should be realistic and should take into account other commitments such as exam periods, holidays, etc.
% Inclusion of a Gantt chart is strongly recommended as a visual representing the project schedule.

Although a general literature review in the field of semantic web and ontology matching have been carried out, as well as an investigation over the tools and software systems currently available, it is observed that a more comprehensive and thorough knowledge about the basics should be equipped due to the considerable theoretical concepts this project is going to relate. Therefore, the project starts with a learning phase where fundamental ideas including description logic, semantic web techniques and ontologies with rules are explored into details in their hierarchical order. The collection of suitable tools and investigation into the relevant APIs are carried along with the learning phase. Then a tangible framework that utilises a selection of currently existing strategies, reasoners and the APIs to automate the process will be implemented and evaluated, which should be delivered with the interim report.
\\\\
The next stage formalises the research into the area of ontology matching, specifically the matching of ABoxes, by alternating and extending the existing approaches to facilitate the discovery of an idea tailored for the need of data integration. The idea will then be validated, implemented, tested, evaluated, and go through a reasoning process if possible, where the results will be concluded in the final dissertation. A Gantt chart with specified stages, tasks, time arrangements and milestones is created with respect to the Waterfall methodology.
\\\\

\begin{figure}[htbp]
\includegraphics[width=\textwidth]{img/Gantt.pdf}
\caption{Gantt Chart}
\label{fig:1}
\end{figure}

\begin{enumerate}[itemsep=0.1em,label=\Alph*.]
	\item Write the project proposal and complete the ethics form.
	\item General review of recent literatures, available tools and software systems.
	\item Gathering of learning materials, advanced literatures, tools and systems.
	\item Comprehensive study of description logic, specifically focusing on the specific language that forms the basis of OWL 2.
	\item Comprehensive study of semantic web techniques, including RDF, OWL 2, rules, SPARQL and RuleML.
	\item General study of ontology engineering strategies and algorithms.
	\item Implementation of the framework, utilising APIs, algorithms and reasoning systems.
	\item Preliminary testing and evaluation of the framework.
	\item Write the Interim Report.
	\item Update the implementation with respect to the feedback. Work load is reduced due to exam preparations.
	\item Systematic investigation into ontology matching techniques, specifically focusing on ABox matching strategies.
	\item Preliminary implementation of the tailored idea for data integration.
	\item Validate idea with supervisor, obtain feedback.
	\item Extend the implementation with the shaped idea and evaluation framework.
	\item Systematic adjustment, testing and evaluation of the implementation with continuous feedback. Reasoning will be carried out if possible.
	\item Write dissertation final report.
	\item Prepare for demonstration.
\end{enumerate}


% Bibliography(0.5 page): Containing some key publications that are either explicitly referred to in the text \cite{Li2017}.

\addcontentsline{toc}{chapter}{Bibliography}
\bibliographystyle{acm}
\bibliography{proposal}


% \begin{appendices}

% e.g., User Manuals, supporting evidence for claims made in the main part of the dissertation (e.g. a copy of a user evaluation questionnaire), samples of test data, etc. Note that Appendices are optional.

\chapter{Artefacts}

There are five items in the submitted artefact folder. Three of them are directories, containing the Jonto Java archive, the LogMap Java archive, and the testing datasets respectively. A user manual named \texttt{MANUAL.md} is attached, which contains detailed instructions for running the tests, and a test script named \texttt{testscript.sh} is provided. A copy of the user manual is presented here for ease of reference.

\section{User Manual}\label{user-manual}

\subsection{Introduction}\label{introduction}

This user manual provides a comprehensive set of instructions for
building and deploying Jonto and LogMap for comparative testing. The
structures of artefacts are introduced, and common pitfalls in running
the tests are explained in detail.

\subsubsection{Jonto}\label{jonto}

The deliverable artefact of Jonto is a single Java archive
\texttt{Jonto.jar}. It runs on either built-in or user-specified
datasets, outputs an alignment in OWL format for all possible entities
to a user specified directory, and evaluates the result with respect to
the reference alignment. Please note that Jonto was built on Java SDK
16, and a lower version of Java may fail to run the program.

\subsubsection{LogMap}\label{logmap}

The deliverable artefact of LogMap is the full version of LogMap 4,
which participated in the 2020 OAEI Campaign in the Anatomy and Largebio
tracks. While it has modes for user interaction and so on, this archived
version was configured to run fully automatically on commandline. It
takes two ontologies and outputs alignments in different formats. It
also evaluates its result with respect to a reference alignment.

The \texttt{LogMap} directory in the artefacts contains the following
items:

\begin{enumerate}
\def\labelenumi{\arabic{enumi}.}
\item
  \texttt{logmap-matcher-4.0.jar} the Java archive of LogMap built using
  Maven.
\item
  \texttt{java-dependencies} the dependencies required by the Java
  archive.
\item
  \texttt{parameters.txt} default parameters used by LogMap.
\end{enumerate}

\subsection{Using the test script}\label{using-the-test-script}

\subsubsection{Running Jonto}\label{running-jonto}

Before running Jonto, it is important to note that the artefact cannot
be put in a directory with spaces and special characters in its path.
Also, please note that the 2-argument version of the command currently
fails for tha Java archive due to a Maven issue.

Jonto can be configured to run matching tasks over either built-in
ontologies or user-specified ones. There are two built-in ontology sets,
being the Anatomy track and FMA\_NCI\_SMALL track as stated in the
report. To run Jonto over these datasets, two commandline arguments need
to be specified. The full command looks like:

\texttt{java -{}-add-opens java.base/java.lang=ALL-UNNAMED -jar \textless{}path\_to\_jonto\_archive\textgreater{} \textless{}dataset\_id\textgreater{} \textless{}absolute\_output\_path\textgreater{}}

Detailed explanations are given below.

\begin{enumerate}
\def\labelenumi{\arabic{enumi}.}
\item
  The \texttt{-{}-add-opens java.base/java.lang=ALL-UNNAMED} VM options
  must be added. This is due to an exceptional attempt to access Java
  internal classes in the OWL API implementation.
\item
  Specify \texttt{\textless{}path\_to\_jonto\_archive\textgreater{}} as
  the path to the \texttt{Jonto.jar} archive. The following 2 segments
  are commandline arguments for Jonto.
\item
  The first commandline argument
  \texttt{\textless{}dataset\_id\textgreater{}} is the ID of the
  built-in dataset to be used. It can be either 0 or 1 at the moment,
  with 0 specifying the Anatomy dataset, and 1 specifying the
  FMA\_NCI\_SMALL track.
\item
  The second commandline argument
  \texttt{\textless{}absolute\_output\_path\textgreater{}} is the output
  path for the generated alignment. It should be absolute and must
  exist. For example: \texttt{/Users/jizhou/Downloads/out/}.
\end{enumerate}

To run Jonto over external datasets, four commandline arguments need to
be specified. The full command looks like:

\texttt{java -{}-add-opens java.base/java.lang=ALL-UNNAMED -jar \textless{}path\_to\_jonto\_archive\textgreater{} \textless{}path\_to\_owl\_1\textgreater{} \textless{}path\_to\_owl\_2\textgreater{} \textless{}path\_to\_reference\_rdf\_or\_txt\textgreater{}\\\textless{}absolute\_output\_path\textgreater{}}

\begin{enumerate}
\def\labelenumi{\arabic{enumi}.}
\item
  The first and second commandline arguments are the two OWL ontologies
  to be matched. They can be in either relative or absolute paths. It is
  important for Jonto that they are given in the correct order as
  specified in the reference alignment.
\item
  The third commandline argument is the reference alignment. It can be
  in either relative or absolute paths, and the RDF and TXT formats are
  supported.
\item
  The fourth commandline argument is the output path for the generated
  alignment. It should be absolute and must exist. For example:
  \texttt{/Users/jizhou/Downloads/out/}.
\end{enumerate}

\subsubsection{Running LogMap}\label{running-logmap}

To run LogMap for evaluation, 6 commandline arguments need to be
specified. The full command generally looks like:

\texttt{java -{}-add-opens java.base/java.lang=ALL-UNNAMED -jar \textless{}path\_to\_logmap\_archive\textgreater{} EVALUATION file:\textless{}absolute\_path\_to\_owl\_1\textgreater{} file:\textless{}absolute\_path\_to\_owl\_2\textgreater{}\\\textless{}absolute\_path\_to\_reference\_rdf\textgreater{} \textless{}absolute\_output\_path\textgreater{} \textless{}classify\_boolean\textgreater{}}

Detailed explanations are given below.

\begin{enumerate}
\def\labelenumi{\arabic{enumi}.}
\item
  The \texttt{-{}-add-opens java.base/java.lang=ALL-UNNAMED} VM options
  must be added. Again, this is due to an exceptional attempt to access
  Java internal classes in the OWL API implementation.
\item
  Specify \texttt{\textless{}path\_to\_logmap\_archive\textgreater{}} as
  the path to the \texttt{logmap-matcher-4.0.jar} archive. The following
  6 segments are commandline arguments for the LogMap matcher.
\item
  The first commandline argument \texttt{EVALUATION} tells LogMap to
  perform evaluation with respect to a reference mapping file.
\item
  The second and third commandline arguments are the two OWL ontologies
  to be matched. LogMap requires them to be specified as a
  \texttt{file:} indicator followed by an absolute path. For example:

  \texttt{file:/Users/jizhou/Downloads/LargeBio\_dataset\_oaei/\\oaei\_NCI\_small\_overlapping\_fma.owl}
\end{enumerate}

Please note that non-ASCII characters as well as some reserved
characters including spaces are not permitted in the OWL ontology path.
If you encounter an \texttt{Illegal character in path at index \#}
error, try to avoid using special characters in the path.

\begin{enumerate}
\def\labelenumi{\arabic{enumi}.}
\setcounter{enumi}{4}
\item
  The fourth commandline argument is the reference alignment RDF. It
  should be an absolute path to an RDF document. For example:

  \texttt{/Users/jizhou/Downloads/LargeBio\_dataset\_oaei/\\oaei\_FMA2NCI\_UMLS\_mappings\_with\_flagged\_repairs.rdf}
\item
  The fifth commandline argument is the output path. It should be
  absolute and must exist. For example:

  \texttt{/Users/jizhou/Downloads/out/}
\item
  The sixth commandline argument is a boolean telling LogMap whether to
  classify the input ontologies together with the mappings or not.
\end{enumerate}


\chapter{Meeting Minutes}

As stated in Section 6.1.3, project related resources are managed through Git version control. The meeting minutes, for example, can be found at \url{https://github.com/Jizhou-Che/Dissertation/blob/master/Log/Minutes.log}. To provide evidence of my orienting role in the meetings and regular progress, however, the full meeting minutes throughout the project are presented below.
\\

\begin{spacing}{1.0}

\lstset{
language={},
basicstyle=\ttfamily\scriptsize,
showspaces=false,
showtabs=false,
breaklines,
tabsize=2
}

\begin{lstlisting}
2020.06.19, Meeting 000 {
	Contents {
		Introduction {
			Ontologies: common vocabulary, TBox, ABox;
			Define rules within ontologies: semantic web rule language (OWL + RuleML);
		}
		Tools {
			Protege with plugins: visualiser to description logic;
			Pellet: reasoners;
		}
		Discussion of the goal of the project;
	}
	Tasks {
		Recent survey papers literature review: focus on summary of the field of ontology matching;
		Explore existing open-source softwares;
	}
}

2020.10.07, Meeting 001 {
	Contents {
		Provide meeting agenda, keep meeting records;
		Project proposal {
			Official deadline: 10.23;
			Supervisor deadline: 10.19;
			Structure {
				Literature review;
				Initial design;
				Implementation;
				Evaluation + evidence;
				Conclusion;
			}
			Question: How will I solve the problem?
		}
		Ethics requirements;
		Dataset: query-able, tables in database, excel, defined in OWL;
		Querying dataset: SPARQL;
		Description logic reasoners: reasoning in OWL;
		Define rules within ontologies: semantic web rule language (OWL 2 + RuleML);
		Knowledge Representation and Reasoning: Dr. Natasha Alechina: http://www.cs.nott.ac.uk/~psznza/G53KRR/;
		Semantic web basics: http://linkeddatatools.com/semantic-web-basics;
		Ontology matching: http://www.ontologymatching.org/publications.html;
		Ontology example: https://doi.org/10.5518/190;
	}
	Tasks {
		Set project focus {
			TBox matching;
			ABox matching;
			Dataset integration;
		}
		Fuzzy logic, rules and ontologies;
		Ontology alignment;
	}
}

2020.10.14, Meeting 002 {
	Contents {
		Project focus {
			Use/Combine existing TBox matching methods;
			Derive/Construct new ABox matching methods;
			Implement data integration as evaluation/reasoning;
		}
		Research area {
			Linked data & semantic web;
			The person who proposed this: Tim Berners-Lee;
			Linked data: same meaning;
			Semantic web: top conference: International Semantic Web Conference (ISWC);
		}
		Dataset source {
			Computer Science Bibliography: dblp: https://dblp.org;
			ISWC on dblp: https://dblp.org/db/conf/semweb/index.html;
			Journal of semantic web (editor: Ian Horrocks): https://www.journals.elsevier.com/journal-of-web-semantics;
			Semantic Web for Earth and Environment Technology Ontology: SWEET Ontology Representation: http://sweetontology.net/repr;
			NCBO BioPortal: https://bioportal.bioontology.org/;
			SWEET on NCBO BioPortal: https://bioportal.bioontology.org/ontologies/SWEET;
			The Environment Ontology: https://sites.google.com/site/environmentontology/Browse-EnvO;
			Research Data Leeds Repository: https://archive.researchdata.leeds.ac.uk;
			UK open data: https://data.gov.uk;
			Ian Horrock's group in Oxford: reasoners;
			Uli Sattler {
				University of Manchester;
				Description logic: An Introduction to Description Logic (book);
				http://www.cs.man.ac.uk/~sattler/;
			}
		}
	}
	Tasks {
		Heshan Du: A Logic of Directions;
	}
}

2020.10.21, Meeting 003 {
	Contents {
		Feedbacks on the draft proposal {
			Clearly state the research question;
			Correctness of contents;
			Academic writing style;
		}
	}
	Tasks {
		Description logic;
		Heshan Du: A Logic of Directions;
		Open Data Nottingham: https://www.opendatanottingham.org.uk/;
	}
}

2020.10.28, Meeting 004 {
	Contents {
		IDEAS {
			Introduce;
			Define;
			Examine;
			Assess;
			Summarise;
		}
		Prioritise the writing;
		Keep notes and BibTeX for readings;
		Office hour of Heshan Du: 14-15 Mon and 10-11 Tue;
		OpenStreetMap: data in Shapefile;
		Geofabric;
	}
	Tasks {
		Pellet reasoner: download source code;
		Protege plugins;
		Produce toy data that can be worked out manually;
		Reference letter: provide supervisor with stamped transcript, CV, PS, and list of schools;
	}
}

2020.11.04, Meeting 005 {
	Contents {
		S-Match: a software for visualisation of ontology matching;
	}
	Tasks {
		Write a program that translates BibTeX into assertional form;
	}
}

2020.11.11, Meeting 006 {
	Contents {
		OWL Primer: https://www.w3.org/2007/OWL/wiki/Primer;
		Ontology Alignment Evaluation Initiative (OAEI): http://oaei.ontologymatching.org;
		Matching properties between assertions;
		Reference Management Tools: Mendeley Reference Manager;
	}
	Tasks {
		Find Instance Matching datasets on OAEI;
		OWL API: https://github.com/owlcs/owlapi/wiki;
	}
}

2020.11.18, Meeting 007 {
	Tasks {
		Read documentation of OWL API;
		Find and design testing dataset;
		Implement framework;
	}
}

2020.11.25, Meeting 008 {
	Contents {
		Feedback on project proposal;
		Discussion on interim report;
		Interim report {
			Outline of project with progress update;
			Structure {
				Introduction;
				Motivation;
				Related work;
				Description of the work methodology;
				Design;
				Implementation;
				Progress {
					Project management {
						Review of work plan;
						Resource and time management;
						Adjustment on work plan;
						Gantt chart;
					}
					Contributions and reflections {
						Details of achievements up to date;
						A personal reflection on the plan and your experience of the project;
						A critical appraisal of how the project has been progressing;
					}
				}
				Bibliography;
			}
			Marking rubric;
		}
	}
	Tasks {
		Read LogMap documentations and papers;
		Investigate ontology matching techniques: tutorial, book;
		Design instance matching structure;
		Investigate evaluation platforms: SEALS, HOBBIT;
		Read documentation of Alignment API;
		Interim Report;
	}
}

2020.12.02, Meeting 009 {
	Contents {
		LogMap documentation;
		Geospatial objects matching;
	}
	Tasks {
		Interim report {
			Official deadline: 12.11;
			Supervisor deadline: 12.06;
			Focus: literature review, design and implementation methodology;
		}
	}
}

2020.12.09, Meeting 010 {
	Contents {
		Interim report {
			Marking rubric;
			Compile as soon as possible;
			State research very clearly in a sentence;
			Related works {
				Focus on instance matching;
				Use simplified ontology matching classification to classify instance matching methodologies;
				API general description;
			}
			Methodology {
				Proposed methodology;
				Input-output details;
				Data;
				Justification: evaluation results;
			}
			Implementation {
				Programming language;
				Tools;
				APIs: detailed usage;
				Reasoning;
				Evaluation framework: detailed usage;
				Compromises;
			}
			Put meeting minutes in appendix;
		}
	}
}

2020.12.16, Meeting 011 {
	Contents {
		No meetings in exam period and vacation;
		Lily system: check reference list in evaluation paper;
	}
	Tasks {
		Start from small fragments of implementation;
	}
}

2021.03.04, Meeting 012 {
	Contents {
		Feedback on interim report;
		Framework design verification {
			Try applying description logic reasoners on reading role axioms;
			Try applying description logic reasoners on mapping repair;
		}
		Project components {
			Class diagram;
			Documentation: not necessary;
			Testing: not necessary;
		}
		Process of final presentation;
		Time management;
	}
}

2021.03.18, Meeting 013 {
	Contents {
		Implementation {
			Read annotations and data properties of instances;
		}
		Final report structure {
			Abstract: very short, summarise main contribution;
			Introduction {
				Original motivation section;
				Description of work;
				Example from research paper;
			}
			Methodology {
				Original design section;
				Compare with LogMap: main difference and contribution;
				Design of algorithms;
				Design of software;
				On an abstract level;
			}
			Implementation {
				Source code fragments: important contributions;
				Reference to used resources: data, library, code, charts;
				On a detailed level;
			}
			Evaluation {
				Selection of datasets: justify;
				Reference to existing experimental results: put tables and references;
				Present the results: improvements, limitations;
				Show understanding;
			}
		}
		
	}
}

2021.03.25, Meeting 014 {
	Contents {
		Project {
			Include a user manual or README with the source code;
		}
		Time management {
			Start writing final report;
		}
	}
}

2021.04.02, Meeting 015 {
	Contents {
		Implementation {
			Generate axioms under assumptions (e.g., disjoint siblings) for testing ontologies, to facilitate DL reasoning: Reading #050;
		}
		Practical {
			Video recording {
				Clear structure of slides;
				Good introduction: with examples;
				Steps of matching: main ones, demonstrate with graphs;
				Run Protege on matching results;
			}
			Artefacts {
				Compiled jar file;
				Shell script to run tests;
				Testing ontologies;
			}
			Showcase brochure {
				Project title, keywords, abstract, images;
			}
		}
		Time management {
			Send fragments of final report to supervisor before meeting;
		}
	}
}

2021.04.08, Meeting 016 {
	Contents {
		Implement precision and confusion matrix: code to calculate automatically;
		Implement running time calculation;
		Compare results with other matching tools such as Lily;
	}
	Tasks {
		Deliver the Introduction and Methodology drafts of final report;
	}
}

2021.04.15, Meeting 017 {
	Contents {
		Final report final structure {
			Title: An Ontology Matching System for Data Integration;
			Abstract: data integration -> ontologies -> ontology matching -> instance matching -> this project -> results;
			Introduction {
				Background;
				Motivation {
					Introduce syntax of description logic and ontologies;
					Justify the help of properties, annotations and role axioms;
					Example of what reasoners can do, and justify the help;
				}
				Summarise description of work briefly: high level aims and objectives;
			}
			Related work {
				All the preparations I did before implementation: description logic, OWL syntax, ontologies, APIs;
				More about available reasoners and APIs;
				Knowledge not restricted to used ones;
			}
			Methodology {
				Description of work in detail: LogMap adaptation, difference and contribution;
				Overall design in abstract level;
				Algorithm design: provide pseudo code;
				Software design: package structures, use of APIs;
			}
			Implementation {
				Build instructions: user manual and README;
				Overall design in detailed level;
				Detailed use of libraries and existing code: provide code fragments if possible;
				Code fragments for core functionalities: indexing extra information, axiom deduction, mapping discovery and repair;
			}
			Evaluation {
				Source and quality of testing ontologies and golden standards: provide links;
				Evaluation results {
					Against LogMap in schema matching;
					Against golden standard in instance matching;
					Against PARIS and Lily in schema and instance matching;
				}
				Future work: further improvements, limitations, solutions;
			}
			Reflection {
				Challenges: available ontologies for instance matching, format of golden standards;
				What I have learned, what I have done well, what I can do better;
			}
			Appendices {
				Meeting minutes;
				Links to repository;
				User manual;
			}
		}
	}
	Tasks {
		Evaluation;
		Final report;
	}
}

2021.04.21, Meeting 018 {
	Contents {
		Artefacts {
			Archive of project and testing systems;
			Test ontologies;
			Test script;
			User manual;
		}
	}
}
\end{lstlisting}
\end{spacing}

\end{appendices}


\end{document}
